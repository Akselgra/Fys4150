% REMEMBER TO SET LANGUAGE!
\documentclass[a4paper,12pt,norsk]{article}
\usepackage[utf8]{inputenc}
% Standard stuff
\usepackage{amsmath,amsthm, amssymb,graphicx,varioref,verbatim,amsfonts,geometry,esint,url}
% colors in text
\usepackage[usenames,dvipsnames,svgnames,table]{xcolor}
% Hyper refs
\usepackage[colorlinks]{hyperref}
\usepackage{float}
\usepackage{wrapfig}
\usepackage{multicol}

\usepackage[export]{adjustbox}

\usepackage{subfig}

% Document formatting
\setlength{\parindent}{0mm}
\setlength{\parskip}{1.5mm}

%Color scheme for listings
\usepackage{textcomp}
\definecolor{listinggray}{gray}{0.9}
\definecolor{lbcolor}{rgb}{0.9,0.9,0.9}

\usepackage{listings}
\lstset{
	backgroundcolor=\color{lbcolor},
	tabsize=4,
	rulecolor=,
	language=python,
        basicstyle=\scriptsize,
        upquote=true,
        aboveskip={1.5\baselineskip},
        columns=fixed,
	numbers=left,
        showstringspaces=false,
        extendedchars=true,
        breaklines=true,
        prebreak = \raisebox{0ex}[0ex][0ex]{\ensuremath{\hookleftarrow}},
        frame=single,
        showtabs=false,
        showspaces=false,
        showstringspaces=false,
        identifierstyle=\ttfamily,
        keywordstyle=\color[rgb]{0,0,1},
        commentstyle=\color[rgb]{0.133,0.545,0.133},
        stringstyle=\color[rgb]{0.627,0.126,0.941}
        }
        
\DeclareMathOperator{\dist}{dist}
\newcommand{\distx}{\dist x}
        
\newcounter{subproject}
\renewcommand{\thesubproject}{\alph{subproject}}
\newenvironment{subproj}{
\begin{description}
\item[\refstepcounter{subproject}(\thesubproject)]
}{\end{description}}

%Lettering instead of numbering in different layers
%\renewcommand{\labelenumi}{\alph{enumi}}
%\renewcommand{\thesubsection}{\alph{subsection}}

%opening

\title{Fys4150 Project 1}
\author{Aksel Graneng}

\begin{document}
\maketitle

\abstract{enter abstract here}

\section{Introduction}

\section{Theory}

\subsection*{Vectorized second derivative}

	If we have a 1d data-set on the form:
	$$\vec{V(x)} = [v_0,\ v_1,\ \cdots,\ v_{n-1},\ v_{n},\ v_{n-1},\ \cdots] $$\\
	Then we can write the second derivative of the data-set as:
	$$f_n = -\frac{v_{n+1} + v_{n-1} - 2v_n}{\Delta x^2} $$\\
	Where $\Delta x$ is the change in variable we are derivating based on; usually time.\\
	\\
	Rather than calculating the second derivatives of this data-set individually, we can instead calculate them all at the same time using linear algebra.\\
	This can be done by finding a matrix $A$ such that
	$$A\vec{V} = \vec{f} $$\\
	Where:
	\begin{gather*}
	\vec{f} = \left[
	\begin{array}{c}
	2v_i - v_2\\
	-v_1 + 2v_2 - v_3\\
	\vdots\\
	-v_{n-1} + 2v_n -v_{n+1}\\
	-v_n + 2v_{n+1} - v_{n+2}\\
	\vdots
	\end{array}
	\right]
	\end{gather*}
	As multiplying $\vec{f}$ with $\frac{1}{\Delta x^2}$ would give us an array containing all the second derivatives. We can see that $\textbf{A}$ must be:
	\begin{gather*}
	\textbf{A} = \left[
	\begin{array}{ccccc}
	2 & -1 & 0 & 0 & \cdots\\
	-1 & 2 & -1 & 0 & \cdots\\
	0 & -1 & 2 & -1 & \cdots\\
	\vdots & 0 & -1 & 2 & -1
	\end{array}
	\right]
	\end{gather*}
\section{Method}

\section{Results}

\section{Discussion}

\section{Conclusion}


\end{document}